%%%%%%%%%%%%%%%%%%%%%%%%%%%%
%%%%%% SERPENT  INTERFACE  %%%%%%
%%%%%%%%%%%%%%%%%%%%%%%%%%%%

\subsection{SERPENT Interface}
\label{subsec:serpentInterface}
The Serpent interface is meant to run multiple user-defined SERPENT simulations
while storing the following values: 
\begin{itemize}
\item Input composition
\item Depletion time
\item Output composition
\item Beginning Of Cycle / End Of Cycle Keff
\end{itemize}

It allows users to run SERPENT by varying a value
(e.g. depletion time, input composition) from a template input file.
For example, a user can run a \texttt{MultiRun} where the depletion
time of a fixed input composition is varied, to see the effect of
depletion time on the output composition and End Of Cycle Keff.

There are some limitations for this interface:
\begin{itemize}
\item Only the parameters mentioned above will be stored in the output csv file
\item You cannot run multiple depletion steps
\end{itemize}

Due to the large number of isotopes that might be tracked,
a utility script is provided in
$RAVEN\_HOME/framework/CodeInterfaces/SERPENT/utilityForCreatingVariables$
to create the variable names that can be used in a RAVEN input file.

The user can edit the \texttt{isoFile} so that it
matches the exact isotope nomenclature used in the SERPENT simulations for the isotopes
to be in the output csv file. To do so:

\begin{enumerate}
\item edit \texttt{isoFile} to match isotope nomenclature in SERPENT
\item run \texttt{python generateCustomVariable.py}
\end{enumerate}

This will automatically generate the feature and target space XML files
used in the RAVEN input file.


\subsubsection{Models}
In order to run the code, make sure you have a valid SERPENT input file.

In the RAVEN input file, your \xmlNode{Models} node should look like:
\begin{lstlisting}[style=XML]
<Models>
    <Code name="SERPENT" subType="Serpent">
      <!-- path to your serpent executable -->
      <executable>/my/path/to/serpent/sss2</executable>
      <clargs arg="" extension=".serpent" type="input"/>
      <clargs arg="--noplot" type="postpend"/>
      <traceCutOff>1e-6<traceCutOff/>
      <isotope_list>mylist.csv<isotope_list/>
      <!-- or 
      <isotope_list>
        1001,290750,34078,38089,411090,451121,
        481191,501281,531390,571450,611570,671661,882280
      <isotope_list/>
      -->
    </Code>
</Models>
\end{lstlisting}
where the \xmlNode{executable} and \xmlNode{clargs} should me modified
to create the appropriate run command for SERPENT. In this example, the command
to run raven is:
\begin{lstlisting}[language=bash]
/my/path/to/serpent/sss2 [inputFile] --noplot
\end{lstlisting}
where the \texttt{inputFile} is defined in the \xmlNode{Files} node as:
\begin{lstlisting}[style=XML]
<Files>
    <Input name="originalInput" type="">serpent_input.serpent</Input>
</Files>
\end{lstlisting}

Two additional XML nodes can be added in the bloc:

\begin{itemize}
   \item \xmlNode{traceCutOff}, \xmlDesc{float, optional parameter},  the trace cut off density.
   \item \xmlNode{isotope\_list}, \xmlDesc{string or comma separated list, required parameter}, This node contains the list
   of isotopes to track. If a ``csv'' is provided, the list is read from such file, otherwise the isotopes are read from the node
   directly.
\end{itemize}

\subsubsection{Files}
The only input file needed is a complete SERPENT input file,
which means that it should either be self sufficient, or includes
the necessary files (e.g. geometry, material definition files).
The input file extension will depend on the the extension definition
in the \xmlNode{Models} node.

\subsubsection{Samplers / Optimizers}
In the \xmlNode{Samplers} block the user can define the variables
to be sampled.
\textbf{Example:} If the user wants to vary depletion time from
10 to 100 days, the SERPENT input file \texttt{dep} definition is as such:
\begin{lstlisting}
dep daystep 
$RAVEN-deptime$ 
\end{lstlisting}

Then in the RAVEN input file, the \xmlNode{Samplers} block would be:
\begin{lstlisting}[style=XML]
  <Samplers>
    <Grid name="myGrid">
      <variable name="deptime">
        <distribution>timedist</distribution>
        <!-- equally spaced steps with lower and upper bound -->
        <grid construction="equal" steps="100" type="CDF">0.1 1</grid>
      </variable>
    </Grid>
  </Samplers>
\end{lstlisting}
where the distributiong \texttt{timedist} is defined in the \xmlNode{Distributions}:
\begin{lstlisting}[style=XML]
  <Distributions>
    <!-- uniform distribution from 0.1 to 1 -->
    <Uniform name="timedist">
      <lowerBound>0.0</lowerBound>
      <upperBound>100</upperBound>
    </Uniform>
</Distributions>
\end{lstlisting}

Then this run would be defined in the \xmlNode{Steps} block as \xmlNode{MultiRun}:
\begin{lstlisting}[style=XML]
<MultiRun name="runGrid">
      <!-- runGrid runs serpent by the number of steps with sampled variable -->
      <Input   class="Files"       type=""          >originalInput</Input>
      <Model   class="Models"      type="Code"      >SERPENT</Model>
      <Sampler class="Samplers"    type="Grid"      >myGrid</Sampler>
      <Output  class="DataObjects" type="PointSet"  >outPointSet</Output>
</MultiRun>
\end{lstlisting}

\subsubsection{Output Files Conversion}
A single SERPENT run creates at least three output files,
\texttt{[input]\_res.m, [input].out, [input]\_dep.m}.
Since the interface is
focused on acquiring the depleted composition, the \texttt{[input].bumat[n]}
file is used as well. Examples of the SERPENT output files are replicated below.

\begin{lstlisting}[language={}]
% bumat file
% .bumat0 is initial, .bumat1 is depleted
mat  fuel  7.98978775163891E-02 vol 1.95057E+07
            % isotope    % atomic density
            3006.09c  1.27142536007818E-06
            3007.09c  2.27080541933364E-02
           90232.09c  3.79539634267135E-03
           92233.09c  6.32566047430223E-05
\end{lstlisting}

\begin{lstlisting}[language={}]
% _res.m file
% The file contains all results by the code, including Keff value
% Only the section with criticality eigenvalues are used.

% Criticality eigenvalues:
% [mean standard dev]
ANA_KEFF  (idx, [1:   6]) = [  9.69430E-01 0.00074  1.07409E-01 0.00074  3.01736E-04 0.01757 ];
IMP_KEFF  (idx, [1:   2]) = [  9.69535E-01 0.00030 ];
COL_KEFF  (idx, [1:   2]) = [  9.69943E-01 0.00070 ];
\end{lstlisting}


The interface parses through \texttt{[input]\_res.m} file
to obtain the Beginning Of Cycle and End Of Cycle Keffs, and the 
\texttt{[input].bumat0} and \texttt{[input].bumat1} file to obtain the
input / output compositions. In the output csv file, the input compositions
are prepended with the letter `f', and the output compositions with the letter `d'
(e.g. f92235, d94239), for separation. 

An example is given below where only two isotopes, U-235 and Pu-241 are tracked.
The values under the isotopes are in atomic density.

\begin{table}[h]
  \centering
  \resizebox{\textwidth}{!}{
    \begin{tabular}{|l|l|l|l|l|l|l|}
      \hline eocKeff
      f92235 & f94241 & deptime & eocKeff & bocKeff & d92235 & d94241 \\
      \hline
      6.32e-05 &  0.00 & 1.0 & 0.9694 &  9.6065 & 3.79e-03 & 7.07e-41 \\
      \hline
      6.32e-05 &  0.00 & 10.0 & 0.9694 &  9.536 & 6.32e-05 & 4.50e-37 \\
      \hline
    \end{tabular}
    }
    \caption{Example csv file generated by RAVEN running SERPENT}
    \label{tab:ser-rav}
\end {table}

